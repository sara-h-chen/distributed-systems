\documentclass[11pt]{article}
\usepackage[utf8]{inputenc}
\usepackage[margin=0.75in]{geometry}
\usepackage{layout}
\usepackage{indentfirst}
\usepackage{url}
\setlength{\voffset}{-0.3in}

%opening
\title{Distributed Systems Summative Assignment}
\author{000280744: jrvh15 - Z0966651}

\begin{document}

\maketitle

\section{Part 1}
\subsection{Generic Workflow with Passive Replication} 
In passive replication, all requests made by the client are executed by the 
primary replica, which, upon completion of the request, would then send an 
update message to its backups. The basic principle behind this is that the 
backups do not execute any invocations but instead only apply the changes 
produced by the invocation execution at the primary server. As such, request 
invocations do not need to be coordinated between the backup servers.

\end{document}
